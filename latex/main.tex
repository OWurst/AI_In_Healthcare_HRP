\documentclass[sigconf]{acmart} % Options: sigconf, sigplan, sigchi, etc.

%% === Metadata ===
\title[Notes Drug Extraction]{Drug Dosage Extraction of Pharmacist Notes in MIMIC Using LLMs for Discrepancy Flagging}
\author{Owen Wurst}
\email{owurst@utexas.edu}
\affiliation{%
  \institution{University of Texas}
  \city{Austin}
  \state{Texas}
  \country{USA}
}

\begin{document}

\begin{abstract}
Your abstract goes here. It should briefly summarize the main points and contributions of your paper.
\end{abstract}

\keywords{MIMIC3, Medical NLP, Large Language Models, Llama, Data Annotation, Machine Learning/Deep Learning}

\maketitle

\section{Introduction}
Introduce your topic, explain the motivation, and outline your contributions.

\section{Related Work}
Summarize and cite prior research relevant to your topic.

\section{Methodology}
Describe your approach, algorithms, models, or system design in detail.

\section{Results}
Present experiments, datasets, evaluation metrics, and findings.

\section{Discussion}
Interpret results, discuss limitations, and compare with prior work.

\section{Conclusion and Future Work}
Wrap up the paper and mention possible extensions.

\begin{acks}
Thank you to Dr. Ying Ding and all the TAs, especially Monte Jarvis. Thank you also to gpt4all, \url{https://www.nomic.ai/gpt4all} for having free access to high performing lightweight LLMs that can be hosted locally.
\end{acks}

\bibliographystyle{ACM-Reference-Format}
\bibliography{references}
\nocite{*}

\end{document}
